\documentclass{exam}

\usepackage[utf8]{inputenc}
\usepackage[swedish]{babel}
\usepackage{amsfonts}
\usepackage{amssymb}
\usepackage{fancyhdr}

\fancyhead[LH]{\textit{Pass 13}}
\fancyhead[CH]{\bfseries \Large SI: Diskret Matematik}
\fancyhead[RH]{\textit{Oscar \& Gustav}}
\fancyfoot[LF]{\textit{Maila oss med uppgifter ni vill se på
passen:\\ \bfseries si@chalmers.it}}
\fancyfoot[RF]{\textit{Hitta de gamla passen:\\ \bfseries si.chalmers.it}}
\renewcommand{\headrulewidth}{0.2pt}
\renewcommand{\footrulewidth}{0.2pt}

\begin{document}
\pagestyle{fancy}
\begin{questions}

\question{\bfseries Tentauppgift: Lite funktioner}
\begin{parts}
\part Hur många funktioner finns från $\mathbb{Z}_{6}$ till$\mathbb{Z}_{12}$?
\part Hur många av dessa funktioner är injektiva, surjektiva,
respektive bijektiva?
\end{parts}

\question{\bfseries Tentauppgift: Lite träd}
\begin{parts}
\part Hur många kanter har ett träd med n noder? 

Visa med induktion!

\part Hur många noder finns det på lägsta nivån i ett komplett binärträd med n nivåer? 

Visa med induktion!

\part Hur många noder finns i ett komplett binärträd med
n nivåer? 

Visa med induktion!

\end{parts}
\question{\bfseries Tentauppgift: Lite Fibonacci}

Fibonacciföljden är definerad av: 

$F(1) = 1$

$F(2) = 1$ 

$F(n) = F(n-1) + F(n-2)$ för $n \geq 3$. 

\begin{parts}
\part Visa att två efterföljande Fibonaccital är relativt prima.
\part Vilka Fibonaccital är jämna? Vilka är delbara med 3? Vilka är delbara med 5?
\end{parts}

\question{\bfseries Tentauppgift: Lite Eulercykler}

\begin{parts}
\part Kan du rita den fullständiga grafen med 5 noder $K_{5}$ utan att lyfta
pennan och utan att gå på samma kant flera gånger? Förklara hur ni gör eller varför det
inte går.

\part För vilka naturliga tal n har grafen $K_{n}$ en Eulercykel?

\part Om en fullständig graf inte har någon Eulercykel, kan man ta bort några
kanter och få en graf med en Eulercykel? Hur många kanter isåfall? 
\end{parts}

\question{\bfseries Tentauppgift: Lite summor}
\begin{parts}
\part Beräkna följande summor:

$1^3 + 2^3$ (mod $3$)

$1^3 + 2^3 + 3^3 + 4^3$ (mod $5$)

$1^3 + 2^3 + 3^3 + 4^3 + 5^3 + 6^3$ (mod $7$)

\part Vad kan man säga om: 

$1^3 + 2^3 + \ldots + 100^3$ (mod $101$)?

\part Ställ upp en förmodan angående summan:

$1^3 + 2^3 + \ldots + (n-1)^3$ (mod $n$)

\part Bevisa er förmodan!

\end{parts}
\end{questions}
\end{document}
