%%This is a very basic article template.
%%There is just one section and two subsections.
\documentclass{exam}

\usepackage[utf8]{inputenc}
\usepackage[T1]{fontenc}
\usepackage[swedish]{babel}
\usepackage{amsfonts}

\pagestyle{headandfoot}
\firstpageheader{\textit{Pass 10: Kombinatorik }}{\bfseries \Large SI: Diskret Matematik}{\textit{Jennifer\& Gustav}}
\firstpageheadrule

\firstpagefooter{\textit{si@chalmers.it}}{}{\textit{si.chalmers.it}}
\firstpagefootrule




\begin{document}


\begin{questions}
\vspace{0.25in}

\question{\bfseries Lite repetition: RSA-krypto}


Antag att den publika nyckeln består av pq = 143
 (där p = 11 och q = 13) och e =23.
\begin{parts}

\part
Varför kan man välja e = 23?

\part
Dekryptera meddelandet 20.
\vspace{0.25in}
\end{parts}

\question{\bfseries Diskussionsfråga : Permutationer och Kombinationer  }
\begin{parts}
\part
När ska man räkna med kombinationer, ange formeln för kombinationer och ge ett exempel.
\part
När ska man räkna med permutationer,  ange formeln för permuationer och ge ett exempel.
\end{parts}
\vspace{0.25in}
\question{\bfseries Permutationer och Kombinationer  }


Antag att det är 25 elever på ett SI pass.
\begin{parts}
\part
På hur många sätt kan man välja 4 personer från ett SI pass? 

\part
Om ordningen i hur eleverna väljs spelar roll, på hur många sätt kan välja 4 elever?
\vspace{0.25in}
\end{parts}



\question{\bfseries  Lite klurigare uppgift :Permutationer och Kombinationer}

Man slänger en tärning fem gånger i rad och skriver ner följden av utfall. 
\begin{parts}
\part
Hur många olika följder kan förekomma?

\part
Hur många olika följder där inget tal förekommer två gånger i rad?

\part
Hur många olika följder där inget tal förekommer tre gånger i rad?

\part
Hur många olika följder där inget tal förekommer fyra gånger i rad?

\part
Hur många olika följder där inger tal förekommer mer än en gång?

\part
Ser ni något samband? Kan man utrrycka  svaret på e) på något annat sätt.

\end{parts}

\vspace{0.25in}

\question{\bfseries Klurigt men nyttigt}

Ge tre exempel på kombinatoriska frågor med svar 60, med lösningar. 
Beräkningarna för de olika exempel skall vara olika.



\end{questions}
\end{document}
