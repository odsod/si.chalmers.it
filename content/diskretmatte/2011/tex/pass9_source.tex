\documentclass{exam}

\usepackage[T1]{fontenc}
\usepackage[swedish]{babel}
\usepackage{amsfonts}
\usepackage{amssymb}
\usepackage{fancyhdr}

\fancyhead[LH]{\textit{Pass 9}}
\fancyhead[CH]{\bfseries \Large SI: Diskret Matematik}
\fancyhead[RH]{\textit{Oscar \& Gustav}}
\fancyfoot[LF]{\textit{Maila oss med uppgifter ni vill se p�
passen:\\ \bfseries si@chalmers.it}}
\fancyfoot[RF]{\textit{Hitta de gamla passen:\\ \bfseries si.chalmers.it}}
\renewcommand{\headrulewidth}{0.2pt}
\renewcommand{\footrulewidth}{0.2pt}

\begin{document}
\pagestyle{fancy}
\begin{questions}

\question{\bfseries Lite mer kongruensr�kning p� exponentiella saker}

F�renkla:

\begin{parts}

\part $3^{23} + 4^{17} \mbox{ (mod 19)}$
\part $4^{15} + (5^9)^{-1} \mbox{ (mod 17)}$

\end{parts}

\question{\bfseries Lite kinesisk restsats}

L�s ut x ur f�ljande ekvationssystem:

\begin{parts}

\part $x \equiv 4 \mbox{ }(\mbox{mod } 17) \\ x \equiv 6 \mbox{ }(\mbox{mod }
9)$

\part $x \equiv 1 \mbox{ }(\mbox{mod } 4) \\ x \equiv 2 \mbox{ }(\mbox{mod } 3)
\\ x \equiv 4 \mbox{ }(\mbox{mod } 5)$

\end{parts}

\question{\bfseries Lite om Eulers phi-funktion}

\begin{parts}

\part Givet ett tal x, vad s�ger $\phi(x)$ om talet x?

\part Ber�kna $\phi(8)$.

\part Ber�kna $\phi(13)$.

\part Ber�kna $\phi(14)$.

\part Ber�kna $\phi(18)$.

\part Ber�kna $\phi(97)$.

\part Ber�kna $\phi(99)$.

\part Ber�kna $\phi(121)$.

\part Ber�kna $\phi(1010)$.

\end{parts}

{\itshape Tips: Kolla r�knereglerna f�r $\phi(n)!$}

\question{\bfseries Lite sneaky stuff}

Givet: $n = 77, e = 13, m = 14$

\begin{parts}

\part Ber�kna $\phi(n)$

\part Ber�kna $d \equiv e^{-1} \mbox{ (mod }\phi(n)\mbox{)}$

{\itshape Tips: Anv�nd $ed \equiv 1 \mbox{ (mod
}\phi(n)\mbox{)}$}

\part Ber�kna $m' = m^{d} \mbox{ (mod n)}$

\part Vad har vi nu gjort..??

\end{parts}

\question{\bfseries Lite mer helt enkelt}

Uttryck v�rdet av $\phi(n)$ med hj�lp av m�ngdbyggaren och predikatlogik.
\end{questions}
\end{document}