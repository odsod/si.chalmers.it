\documentclass{exam}

\usepackage[T1]{fontenc}
\usepackage[swedish]{babel}
\usepackage{amsfonts}
\usepackage{amssymb}
\usepackage{fancyhdr}

\fancyhead[LH]{\textit{Pass 5: Mer om induktion}}
\fancyhead[CH]{\bfseries \Large SI: Diskret Matematik}
\fancyhead[RH]{\textit{Oscar \& Gustav}}
\fancyfoot[LF]{\textit{Maila oss med uppgifter ni vill se p�
passen:\\ \bfseries si@chalmers.it}}
\fancyfoot[RF]{\textit{Hitta de gamla passen med lite
svar:\\ \bfseries http://nollk.it/si}}
\renewcommand{\headrulewidth}{0.2pt}
\renewcommand{\footrulewidth}{0.2pt}

\begin{document}
\pagestyle{fancy}
\begin{questions}

\question {\bfseries Lite klurig repetition}

Ange st�rsta, minsta, minimala och maximala element i f�ljande partiella
ordningar:

\begin{parts}
\part $\displaystyle A = \left\{ \{\}, \{1\}, \{2\}, \{1,2\} \right\}$

$R = (A, \subseteq)$

\part $\displaystyle A = \left\{ \{\}, \{1\}, \{2\}\right\}$

$R = (A, \supseteq)$

\part $R = (\mathbb{N}, \leq)$

\part $R = (\mathbb{N}, \geq)$
\end{parts}
\question {\bfseries Lite induktion}

Bevisa att f�ljande egenskaper g�ller f�r $\mathbb{N}$:

\begin{parts}
\part
$\displaystyle \sum_{i=0}^{n}n^3 = \left(\frac{n(n+1)}{2}\right)^2$

\part
$\displaystyle \sum_{k=0}^{n}k \cdot 2^{k-1} = (n-1) \cdot 2^n + 1$
\end{parts}

Tips: Anv�nd induktion!

\question {\bfseries Lite brainfuck}

En schackbr�de best�r av $8 \times 8$ rutor. Hur m�nga kvadrater finns det p�
br�det? 

Kvadrater har lika m�nga rutor p� bredden som p� h�jden: $1 \times 1, 2 \times
2, 3 \times 3$ etc.

Svara f�rst med en summa och ber�kna sedan det exakta antalet.

\question {\bfseries Lite brainfuck!}

Ett generellt rutigt br�de best�r av $n \times m$ rutor. Hur m�nga rektanglar
finns det p� ett s�dant br�de? 

Svara f�rst med hj�lp av summatecken och sedan utan.

\end{questions}

\end{document}
