%%This is a very basic article template.
%%There is just one section and two subsections.
\documentclass{exam}

\usepackage[utf8]{inputenc}
\usepackage[T1]{fontenc}
\usepackage[swedish]{babel}
\usepackage{amsfonts}

\pagestyle{headandfoot}
\firstpageheader{\textit{Pass 12: Grafteori och tentaplugg}}{\bfseries \Large SI: Diskret Matematik}{\textit{Jennifer\& Gustav}}
\firstpageheadrule

\firstpagefooter{\textit{si@chalmers.it}}{}{\textit{si.chalmers.it}}
\firstpagefootrule




\begin{document}


\begin{questions}


\question{\bfseries Grafteori}

Rita och förklara följande begrepp:


\begin{parts}
\part
Graf

\part
Väg

\part
Gradtal

\part
Eulercykel

\part
Hamiltoncykel

\part
Komplettgraf

\part
Bipartit graf

\part
Inducerad delgref

\part
Sammanhängande graf (och maximalt sammanhängande graf)

\part
komponent

\vspace{0.25in}
\end{parts}
\question{\bfseries Lättare tentauppgift}

Ge exempel på en relation som är 
\begin{parts}
\part
en ekivavlensrelation,
\part
transitiv men varken reflexiv eller symmetrisk,
\part
symmetrisk, men varken reflexiv eller transitiv.

\end{parts}
\vspace{0.25in}
\question{\bfseries  Tentauppgift 22 aug 2006  }

Ungefär 2500 enkronor staplas på ett bord. När de läggs i högar om 13 mynt blir det 2 mynt över och när de läggs i högar om 21 mynt blir det 14 mynt över. Hur många mynt är det totalt?
\vspace{0.25in}
\question{\bfseries  Tentauppgift 22 aug 2006  }

Till sin mattetenta köpte Kalle tablettsaskar för 7 kronor styck, klubbor för 3 kronor styck och chokladkakor för 10 kronor styck. Han köpte sammanlagt 26 saker och betalade sammanlagt 175 kronor. Hur många av varje sort köpte han? (Obs. det kan finnas flera lösningar.)

\vspace{0.25in}
\question{\bfseries  Tentauppgift 18 dec 2008  }
\begin{parts}
\part
Visa att p | ${{a}\choose {b}}$  för alla 1 $\leq$ k < p,


\part
Låt a och b vara heltal. Visa att i så fall är $(a+b)^{p} \equiv a^{p} + b^{p} $ (mod p).
(Tips! Använd resultatet i första deluppgiften.)
\end{parts}




\end{questions}
\end{document}
