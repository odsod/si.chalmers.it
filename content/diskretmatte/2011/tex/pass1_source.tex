%%This is a very basic article template.
%%There is just one section and two subsections.
\documentclass{exam}

\usepackage[T1]{fontenc}
\usepackage[swedish]{babel}
\usepackage{amsfonts}
\usepackage{fancyhdr}

\fancyhead[LH]{\textit{Pass 1: Logik}}
\fancyhead[CH]{\bfseries \Large SI: Diskret Matematik}
\fancyhead[RH]{\textit{Oscar \& Gustav}}
\fancyfoot[LF]{\textit{si@chalmers.it}}
\fancyfoot[RF]{\textit{http://nollk.it/si}}
\renewcommand{\headrulewidth}{0pt}

\begin{document}
\pagestyle{fancy}

\begin{questions}

\question{\bfseries Lite predikatlogik och kvantorer}

Betrakta m�ngden av alla paket som ligger under julgranen. 

\begin{parts}

\part
Skriv f�ljande utsagor p� symbolisk logisk form och illustrera dem med hj�lp av
Venndiagram.

\begin{itemize}
  \item Minst ett gult paket �r till mig
  \item Alla gula paket �r tomma
  \item Inget stort paket �r tomt
\end{itemize}

\part
Vad kan du dra f�r ytterligare slutsatser med hj�lp av ovanst�ende utsagor?
\end{parts}

\question{\bfseries Lite tautologier}

Vilka av f�ljande logiska formler �r tautologier?

\begin{parts}
\part $(P \rightarrow Q) \vee (Q \rightarrow P)$
\part $(P \wedge (P \rightarrow Q)) \rightarrow Q$
\part $(P \wedge Q \wedge \neg R) \rightarrow (R \rightarrow \neg (P \wedge Q))$
\end{parts}

Tips: Anv�nd sanningstabeller!

\question{\bfseries Lite utsagor}

Betrakta f�ljande predikat: $Q(x,y,z) : x^2 + y^2 \geq z^2$.

Vilka av f�ljande utsagor �r sanna? Motivera varje svar.
 
\begin{parts}
\part $\forall x : \forall y : \forall z : Q(x,y,x)$
\part $\forall x : \forall y : \exists z : Q(x,y,x)$
\part $\exists x : \exists y : \forall z : Q(x,y,x)$
\part $\forall x : \exists y : \exists z : Q(x,y,x)$
\part $\forall x : \forall y : \exists z : Q(x,y,x)$
\end{parts}

I samtliga fall g�ller: $x,y,z \in \mathbb{R}$.

\question{\bfseries Lite d�ggdjur}

Antag att v�rt universum �r m�ngden av alla djur.

Betrakta f�ljande predikat:
\begin{itemize}
  \item $D(x)$ : x �r ett d�ggdjur
  \item $H(x)$ : x �r en hund
  \item $A(x)$ : x �r en anka
  \item $T(x,y)$ : x �r tyngre �n y
\end{itemize}

Formulera f�ljande p�st�enden med hj�lp av kvantorer och ovanst�ende predikat:

\begin{parts}
  \part Alla hundar �r d�ggdjur
  \part Inga hundar �r ankor
  \part Det finns ett d�ggdjur som �r tyngre �n alla hundar
  \part Det finns en anka som �r l�ttare �n alla hundar
\end{parts}

\end{questions}


\end{document}
